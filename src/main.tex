% Preamble
\documentclass{moderncv}
\moderncvstyle{classic}
\moderncvcolor{blue}

\newif\ifen
\newif\iffr

\newcommand{\en}[1]{\ifen#1\fi}
\newcommand{\fr}[1]{\iffr#1\fi}

\entrue

\fr{\title{Ingénieur en développement web agile passionné}}
\en{\title{Enthusiast software engineer}}
\firstname{Émilien}
\lastname{Boulben}
\mobile{+33 6 60 93 98 10}
\email{emilien.boulben@gmail.fr}
\address{}{92170 Vanves\en{, France}}

\colorlet{formationcolor}{green!55!black}
\colorlet{procolor}{orange!60!black}
\colorlet{associationcolor}{violet!90!white!70!black}

\usepackage{multicol}

\usepackage{enumitem}
\newlist{junior}{itemize}{1}
\setlist[junior, 1]{label=$\diamond$, leftmargin=3em, itemsep = 0em}

\usepackage{geometry}
\geometry{
    left=5mm,
    right=5mm,
    top=5mm,
    bottom=5mm,
    heightrounded,
}

\usepackage{moderntimeline}

\makeatletter
\renewcommand{\tltextend}[2][north east]{%
    \tikzset{
        tl@endyear/.style={
            font=#2,
            name=tl@endyear,
            below=\tl@width - 0.4ex,
            inner xsep=0pt,
            anchor=#1,
        }
    }
}
\renewcommand{\tltextstart}[2][base west]{%
    \tikzset{
        tl@startyear/.style={
            font=#2,
            name=tl@startyear,
            above=\tl@width + 0.6ex,
            inner xsep=0pt,
            anchor=#1,
        }
    }
}
\renewcommand{\tltextsingle}[2][base]{%
    \tikzset{
        tl@singleyear/.style={
            font=#2,
            name=tl@singleyear,
            above=1ex,
            inner xsep=0pt,
            anchor=#1
        }
    }
}
\makeatother

\tlenablemonths
\tltextstart[base]{\footnotesize}
\tltextend[north]{\footnotesize}
\tltextsingle[base]{\footnotesize}
\tlsince{\fr{Depuis }\en{Since }}
\tlwidth{0.8ex}
\tlmaxdates{2009}{2019}
\AtBeginDocument{\setlength{\hintscolumnwidth}{3cm}}

% Document
\begin{document}

    \maketitle

    \setlength{\parskip}{-3em}


    \section{\textbf{\fr{Expérience Professionnelle et associative}\en{Professional and Volunteer Experience}}}\label{sec:experience}

    \setlength{\parskip}{0.7em}

    \tlcventry[procolor]{2016/10}{0}{\fr{Développeur Web Java}\en{Java Web Software Engineer}}
    {Valtech}{Client - Primonial}{}
    {\fr{Développement de projets pour une plateforme de souscription électronique avec une architecture micro-services.
    Accompagnement du métier dans l'expression du besoin et les spécifications. Équipe de 5 à 8 personnes.}
    \en{Development of various projects for an electronic subscription platform, with a micro-services architecture.
    Business support to define user's needs and write specifications. Team of 5 to 8 people.}}

    \tlcventry[procolor]{2015/11}{2016/10}{\fr{Développeur Web Java / Devops}\en{Java Web Software Engineer / Devops}}
    {Valtech}{Client - AFPA}{}
    {\fr{Refonte de l'intégration continue d'un projet existant, implémentation de nouveaux besoins, optimisation des performances (Java et Transact-SQL).
    Mission en autonomie sans équipe.}
    \en{Software factory upgrade of legacy project, new needs development, performance optimisations (Java and Transact-SQL). Autonomous job without team.}}

    \tlcventry[procolor]{2014/8}{2015/10}{\fr{Développeur Web Java}\en{Java Web Software Engineer}}
    {\fr{Apprentissage à Valtech}\en{Valtech apprenticeship}}{Client - Société Générale}{}
    {\fr{Développement de modules web full stack (front vanilla, back Spring) dans une équipe agile, méthode SCRUM.\newline
    Automatisation des tâches en shell. Équipe de 2 à 6 personnes.}
    \en{Full stack development of many Web apps (vanilla front, Spring back), using SCRUM method.\newline
    Improvement of software factory with bash. Team of 2 to 6 people.}}

    \tlcventry[associationcolor]{2013/11}{2014/7}{\fr{Responsable Technique -- Administrateur système}\en{Technical Leader -- System Administrator}}%
    {Junior ISEP}{}{}%
    {
    \begin{junior}
        \begin{multicols}{2}
            \item \fr{Infrastructure informatique Debian}\en{Debian infrastructure}
            \item \fr{Formations~: Linux, git, Java, web\ldots}\en{Training: Linux, git, Java, web\ldots}
            \item \fr{Référent technique pour les missions, ou exécution}\en{Contracts technical leader}
            \item \fr{Résolutions de litiges clients}\en{Customer disputes resolution}
        \end{multicols}
    \end{junior}
    }

    \tlcventry[procolor]{2013/9}{2014/7}{\fr{Développeur C++ en R\&D}\en{C++ R\&D Software Developer}}
    {}{\fr{Apprentissage à Sodern}\en{Sodern apprenticeship}}{}
    {\fr{Création d'un logiciel d'étude de tâches de diffraction en C++ avec la bibliothèque ROOT du Cern.}
    \en{Development of a diffraction spots identification software using Cern's ROOT C++ library.}}

    \tlcventry[procolor]{2013/9}{2016/6}{\fr{Khôlleur de Mathématiques}\en{Mathematics Test Preparation Instructor}}
    {ISEP}{MPSI}{}{}

    \tllabelcventry[procolor]{2013/7}{2013/8}{\fr{Été///2013}\en{2013///Summer}}
    {\fr{Dévelopeur Java Core}\en{Java Core Developer}}
    {\fr{Stage à Airbus}\en{Airbus internship}}{Ottobrun (\fr{Allemagne}\en{Germany})}{}
    {\fr{Représentation de signaux en direct dans une interface Java Swing, contraintes fortes de performance.}
    \en{Live signals drawing within a Java Swing interface with big performances constraints}}

    \tlcventry[associationcolor]{2012/10}{2014/1}{ISEPA - \fr{Président Fondateur}\en{Founder and President}}
    {}{Isep Students Exchange \& Partnership Association}{}
    {\fr{Création d'une association pour faciliter les échanges culturels entre tous les élèves de l'ISEP.}\en{Association to ease cultural exchanges between ISEP's students}}

   \tlcventry[associationcolor]{2009/9}{2010/6}{\fr{Chef d’Orchestre de l’Ensemble Musical du Lycée Saint-Louis}\en{School orchestra conductor}}{}
    {}{}{}

    \tltextend[north]{\footnotesize}


    \section{\textbf{\fr{Compétences}\en{Skills}}}\label{sec:competences}

    \setlength{\parskip}{0.0em}
    \cvdoubleitem{\textbf{Back-end}\fr{~}:}{Java, Spring, Jpa, Sql, \LaTeX}{\textbf{Front-end}\fr{~}:}{\fr{bases de html5, css3, Js et angular}\en{html5, css3, JS and angular basics}}
    \cvdoubleitem{\textbf{Devops}\fr{~}:}{git, Linux, bash, Jenkins, maven}{\textbf{Projet}\fr{~}:}{Scrum, Kanban}

    \cvitem{\textbf{\fr{Anglais}\en{English}}\fr{~}:}{\fr{courant}\en{fluent}, TOEIC 900 (2013)}


    \setlength{\parskip}{0.3em}


    \section{\textbf{Formation}}\label{sec:formation}

    \tlcventry[formationcolor]{2012/9}{2015/9}{\fr{Diplôme d’Ingénieur Génie Logiciel en apprentissage}\en{Master of Science in Software Engineering}}{ISEP}
    {}{}{}

    \tlcventry[formationcolor]{2009/9}{2010/6}{\fr{Classe préparatoire}\en{Preparatory school}}{\fr{Lycée Saint-Louis}\en{Saint-Louis School}}
    {\fr{MPSI}\en{Mathematic major}}{}{}

    \tldatecventry[formationcolor]{2009}
    {\fr{Baccalauréat Scientifique option Sciences de l'Ingénieur}\en{High School Degree with Engineering major}}{}%
    {\fr{Mention Bien}\en{With good appreciation}}{}{}{}


    \section{\textbf{\fr{Centres d'intérêts}\en{Hobbies}}}\label{sec:interest}

    \setlength{\parskip}{0.0em}
    \cvitem{\textbf{Sports}\fr{~}:}{\fr{Escalade, alpinisme, ski}\en{Climbing, mountaineering, skiing}}
    \cvitem{\textbf{\fr{Divers}\en{Others}}\fr{~}:}{\fr{lecture, \oe nologie}\en{reading, \oe nology}}
    \cvitem{\textbf{\fr{Musique}\en{Music}}\fr{~}:}{\fr{alto, piano, écriture et arrangements, direction d'orchestre}\en{viola, piano, writing and arrangement, conductor}}

\end{document}
